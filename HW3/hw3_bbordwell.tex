%
%     hw3_bbordwell.tex
%     Baylee Bordwell (baylee.bordwell@colorado.edu)
%     Based on the template by Benjamin Brown (bpbrown@colorado.edu)
%     Aug 27, 2014
%
%     Problem set 3 for ASTR/ATOC 5540, Mathematical Methods, taught at
%     University of Colorado, Boulder, Fall 2014.
%
%

\documentclass[12pt, preprint]{aastex}
% formatting based on 2014 NASA ATP proposal with Jeff Oishi

%%%%%%begin preamble
\usepackage[hmargin=1in, vmargin=1in]{geometry} % Margins
\usepackage{hyperref}
\usepackage{url}
\usepackage{times}
\usepackage{natbib}
\usepackage{graphicx}
\usepackage{amsmath}
\usepackage{amsfonts}
\usepackage{amssymb}
\usepackage{pdfpages}
\usepackage{import}
% for code import
\usepackage{listings}
\usepackage{color}
\usepackage{ragged2e}

\hypersetup{
     colorlinks   = true,
     citecolor     = gray,
     urlcolor       = blue
}

%% headers
\usepackage{fancyhdr}
\pagestyle{fancy}
\lhead{ASTR/ATOC 5540}
\chead{}
\rhead{name: Baylee Bordwell}
\lfoot{Problem Set 3}
\cfoot{\thepage}
\rfoot{Fall 2014}
% no hline under header
\renewcommand{\headrulewidth}{0pt}

\newcommand{\sol}{\ensuremath{\odot}}

% make lists compact
\usepackage{enumitem}
%\setlist{nosep}

%%%%%%end preamble


\begin{document}
\section*{Problem Set 3: Fitting Solar/Stellar data without error}
\begin{enumerate}
\item 
\begin{figure}[!ht]  \centering
  \includegraphics[width=5in]{hw3_fig.png}
\end{figure}

\item $L_x = C * \Phi^p$ $\Rightarrow$ $\ln(L_x) = p\ln(\Phi) + \ln(C)$ \newline
For $n$ unknowns, the system of equations, $A\mathbf{x} = \mathbf{b}$ holds, where for an unknown n $\mathbf{b_n}$ is $L_{x,n}$, $x_1$ is $p$, $x_2$ is $\ln(C)$, $A_{n1}$ is $\ln{\Phi}$ and $A_{n2}$ is 1.

\item The \verb|mflux_lx_all.txt| file is 1316 lines long, so $n = 1316$. $A$ is 2 by $n$, $\mathbf{x}$ is 1 by 2, and $\mathbf{b}$ is 1 by $n$. The system of equations looks like,
\[\left(\begin{array}{cc}
$A_{11}$ & $A_{12}$ \\
$A_{21}$ & $A_{22}$ \\
\vdots & \vdots \\
$A_{n1}$ & $A_{n2}$ \\
\end{array}\right)
\left(\begin{array}{c}
$x_1$ \\
$x_2$ \\
\end{array}\right) = 
\left( \begin{array}{c}
$b_1$ \\
$b_2$ \\
\vdots \\
$b_n$ \\
\end{array}\right)
\]

\item 
\[ \begin{array}{l}
$A\mathbf{x}=\mathbf{b}$ \\
$A^TA\mathbf{x}=A^T\mathbf{b}$ \\
$\mathbf{x}=(A^TA)^{-1}A^T\mathbf{b}$ \\
\end{array}
\] 
The value of $\mathbf{x}$ found above minimizes $(A\mathbf{x}-\mathbf{b})^T(A\mathbf{x}-\mathbf{b})$ because it is the solution to the equation $(A\mathbf{x}-\mathbf{b})=0$ and  $(A\mathbf{x}-\mathbf{b})^T(A\mathbf{x}-\mathbf{b})$ is effectively  ``$(A\mathbf{x}-\mathbf{b})^2$'', so with $\mathbf{x}$ as above $(A\mathbf{x}-\mathbf{b})^T(A\mathbf{x}-\mathbf{b})=0$, which is the minimum value for a real squared number.

\item Solving for x yields: $p = 1.1492$, $C = 8.7255\cdot 10^{-1}$ \newline
My $p$ does agree with Pevtsov et al. (2003).

\item Overplotted in the figure in question 1. The mean absolute error in logspace of the fit is .546. If the mean absolute error was calculated in non-logspace, it would be huge, because the linear fit optimizes the fit to have approximately the same error for all points, and the scale of the error for the points with higher M$_x$ would suddenly be much greater than that of points with lower M$_x$.
\end{enumerate}

\noindent Code involved in this assignment:
\input{hw3_code_bbordwell.tex}
\end{document}
